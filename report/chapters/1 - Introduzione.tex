\chapter{Introduzione}

Il progetto che si intende sviluppare è un'indagine accurata e approfondita sulla libreria \textbf{Cats}, sia dal punto di vista teorico che pratico. Lo scopo principale è quello di evidenziare i motivi per cui la libreria che fornisce astrazioni per linguaggi funzionali nel linguaggio Scala viene utilizzata. Inoltre, si presta particolare attenzione ai meccanismi relativi alla concorrenza, all'I/O e alla gestione delle risorse, utilizzando il supporto di Cats-Effect, una libreria funzionale per la gestione dei side-effect e delle operazioni asincrone.

La libreria in questione offre numerosi vantaggi, come la capacità di gestire in modo efficiente il multithreading e l'asincronia, nonché di garantire la scalabilità delle applicazioni. Questo la rende una scelta ideale per la realizzazione di applicazioni distribuite e altamente concorrenti. Grazie all'utilizzo di Cats-Effect, la libreria offre una grande flessibilità nella gestione delle risorse, permettendo di ottimizzare le prestazioni e di evitare la comparsa di effetti collaterali indesiderati.

Per dimostrare l'utilità e la versatilità di questa libreria, il progetto prevede la presentazione di numerosi esempi di codice, illustrando le funzionalità principali della libreria e mostrando come queste possano essere utilizzate per risolvere problemi comuni nello sviluppo di applicazioni. Verranno condotti esperimenti più approfonditi, che consentiranno di esplorare le funzionalità avanzate della libreria e di comprendere meglio le modalità di utilizzo.

Infine, come parte del progetto, verranno realizzate delle RestAPI con Cats-Effect, dimostrando così come la libreria possa essere utilizzata in modo pratico per lo sviluppo di applicazioni web. Le RestAPI saranno una dimostrazione pratica delle capacità della libreria. In sintesi, il progetto si propone di approfondire la libreria in questione, mostrandone le funzionalità principali e dimostrando come possa essere utilizzata per risolvere problemi comuni e caratteristici nello sviluppo di applicazioni.