\chapter{Conclusioni}

In conclusione mi ritengo molto soddisfatto del lavoro svolto. Posso affermare che sviluppare in contesti concorrenti con un approccio funzionale è possibile e anche pratico grazie a scala, cats e cats-effect. Le librerie forniscono un alto livello di astrazione e permettono forti customizzazioni garantendo lo stile funzionale. In ambito web, per le mie esperienze ritengo che possa trovare i suoi casi d'uso in contesti e scenari dove è necessario gestire in maniera concorrente grandi moli di dati e operazioni, sicuramente seguendo i nuovi trend della programmazione poliglotta spesso supportati anche dai grandi cloud services provider come AWS, potrebbe trovare contesti applicativi per porzioni di programmi che necessitano l'uso di Scala e della programmazione funzionale. Queste considerazioni derivano dal fatto che in un contesto aziendale/reale Scala risulta un linguaggio di nicchia e di alto livello, l'approccio funzionale lo rende ancora più complesso, rendendolo più snello e meno verboso ma questo fa si che di per sè il codice sia meno chiaro e leggibile. Inoltre, in generale Scala ha una curva di apprendimento che inizialmente è molto alta, Cats e Cats Effect malgrado siano ben documentate anche esse hanno una curva di apprendimento molto alta, ci sono moltissime typeclasses e tantissime API, padroneggiarle risulta comunque complesso e dispendioso in termini di tempo. Sempre in merito a queste librerie, usciti dalla documentazione è complicato trovare riferimenti, esempi, use case o informazioni utili non solo per Cats e Cats effect ma anche per http4s-ember; questo mi ha portato a conoscere alcune cummunity italiane di sviluppatori Scala grazie alle quali ho potuto capire di più in merito alle librerie usate e a come vengono usate nei contesti reali. Ritengo comunque tutta questa esperienza importante e molto formativa. Sicuramente Scala e la programmazione funzionale sono aspetti che stanno crescendo molto negli ultimi periodi, forse sono ancora un pò acerbi sotto alcuni punti di vista (grande stacco tra Scala 2 e Scala 3 e ancora non tutte le librerie supportano la nuova versione di linguaggio) ma in generale credo sia un trend che possa portare grandi novità basti pensare a Scala.js e Scala Native.