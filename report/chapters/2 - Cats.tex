\chapter{Cats}

\section{Storia}
La libreria Cats è nata nel 2014, grazie all'iniziativa di un gruppo di sviluppatori di software funzionale, tra cui Michael Pilquist, Travis Brown, Lars Hupel e altri. Essi hanno avvertito la necessità di creare una libreria che potesse fornire un'implementazione consistente e componibile dei concetti fondamentali della programmazione funzionale.

\noindent Cats è stata ispirata dalla libreria Scalaz, una libreria di supporto per la programmazione funzionale in Scala, sviluppata da Tony Morris e dagli altri membri della comunità Scalaz. Tuttavia, rispetto a Scalaz, Cats è stata progettata con un'attenzione maggiore alla modularità e alla compatibilità con altre librerie di Scala.

\noindent Il nome "Cats" è un acronimo di "Category Theory Scala", in riferimento alla teoria delle categorie, una branca della matematica che fornisce un linguaggio formale per la descrizione di concetti astratti e relazioni tra di essi. La teoria delle categorie è stata una fonte di ispirazione per la progettazione di Cats, in quanto essa offre un'astrazione potente e componibile per i concetti fondamentali della programmazione funzionale.

\noindent Negli anni successivi alla sua nascita, Cats ha avuto un crescente successo all'interno della comunità di sviluppatori Scala, diventando una delle librerie più utilizzate per la programmazione funzionale. Grazie alla sua architettura modulare e alla compatibilità con altre librerie di Scala, essa ha consentito lo sviluppo di applicazioni altamente performanti, robuste e scalabili, permettendo di sfruttare al massimo le potenzialità della programmazione funzionale.

\noindent Ad oggi, Cats è una libreria molto utilizzata dalle grandi aziende di sviluppo software che adottano la programmazione funzionale in Scala. Ad esempio, Twitter utilizza Cats all'interno della propria infrastruttura di servizi, sfruttando le funzionalità di concorrenza e di gestione delle risorse per sviluppare applicazioni altamente performanti e scalabili. In particolare, Twitter ha sviluppato una libreria di supporto chiamata "Finagle", che si basa su Cats e su altre librerie di Scala, per fornire un'architettura di servizi distribuiti altamente efficiente e affidabile.

\noindent Anche la società di consulenza tecnologica Netflix utilizza Cats all'interno della propria piattaforma di streaming video, sfruttando le funzionalità di gestione delle risorse e di concorrenza per garantire prestazioni elevate e stabili.

\noindent Oltre alle grandi aziende, anche molte startup e imprese di medie dimensioni utilizzano Cats per lo sviluppo di applicazioni web, servizi backend e strumenti di analisi dati. Grazie alla sua architettura modulare e alla flessibilità, Cats è in grado di soddisfare le esigenze di una vasta gamma di applicazioni e di ambienti di sviluppo.

\noindent Essa fornisce un'ampia gamma di funzionalità e costrutti, che consentono di scrivere codice più conciso, espressivo e robusto permettendo astrazioni per la programmazione funzionale fornendo supporti per la gestione dell'I/O e della concorrenza. Una delle funzionalità principali di Cats è la gestione degli side-effect e delle operazioni asincrone, che è ottenuta attraverso l'utilizzo di tipi di dati funzionali come i Funtori, i Monoidi, le Applicative che approfondiremo in seguito. Questi tipi di dati consentono di gestire i side-effect e le operazioni asincrone in modo sicuro e componibile. Cats offre una serie di costrutti e funzionalità utili per la gestione delle collezioni di dati, tra cui mappe, insiemi, sequenze e stream. Questi costrutti consentono di manipolare le collezioni di dati in modo elegante e funzionale, fornendo numerosi metodi e funzioni per la trasformazione, l'ordinamento e l'aggregazione dei dati. Inoltre, fornisce anche una serie di funzionalità per la gestione delle eccezioni, compresa la gestione degli errori e la loro propagazione in modo sicuro e componibile. La libreria offre un supporto completo per la programmazione generica, consentendo di scrivere codice parametrico e riutilizzabile. Cats contiene un'ampia gamma di funzionalità per la programmazione asincrona, tra cui il supporto per la gestione degli eventi, la gestione delle risorse e la concorrenza. Queste funzionalità consentono di scrivere applicazioni altamente performanti e scalabili, sfruttando al massimo le capacità di Scala e della programmazione funzionale.

\noindent In sintesi, la libreria Cats offre una vasta gamma di funzionalità e costrutti, che consentono di scrivere codice più conciso, espressivo e robusto. Grazie alla sua architettura funzionale, essa è particolarmente adatta per lo sviluppo di applicazioni asincrone, distribuite e altamente concorrenti.

\noindent Le seguenti sezioni mostrano i costrutti matematici di programmazione funzionale forniti da Cats.

\section{Costrutti Matematici}

Prima di entrare nel dettaglio di Cats e successivamente Cats-effect è necessario definire brevemente i principali costrutti matematici che vengono utilizzati soprattutto da Cats: semigroup, monoidi, monadi e funtori. Tutti costrutti devono aderire a delle laws per essere definiti in modo opportuno.

\subsection{Laws}
Concettualmente, tutte le classi di tipo sono dotate di leggi. Queste leggi vincolano le implementazioni per un dato tipo e possono essere sfruttate e utilizzate per ragionare sul codice generico.

\subsection{Semigroup}
Si definisce come:
\begin{minted}{scala}
trait Semigroup[A] {
  def combine(x: A, y: A): A
}
\end{minted}

\noindent Un semigruppo per un certo tipo A ha una singola operazione combine, che prende due valori di tipo A, e restituisce un valore di tipo A. Questa operazione deve essere garantita come associativa. Vale a dire che:
\begin{minted}{scala}
((a combine b) combine c)
\end{minted}
Deve essere uguale a:
\begin{minted}{scala}
(a combine (b combine c))
\end{minted}

\noindent Associatività significa che la seguente uguaglianza deve valere per qualsiasi scelta di x, y, e z.
\begin{minted}{scala}
combine(x, combine(y, z)) = combine(combine(x, y), z)
\end{minted}

\noindent Infatti supponendo di usare una combine di Int avremmo che:
\begin{minted}{scala}
val x = 1
val y = 2
val z = 3
combine(x, combine(y, z)) == combine(combine(x, y), z))
=> true
\end{minted}
\subsection{Monoidi}
Si definisce come:
\begin{minted}{scala}
trait Semigroup[A] {
  def combine(x: A, y: A): A
}

trait Monoid[A] extends Semigroup[A] {
  def empty: A
}
\end{minted}

\noindent Monoid estende la classe di tipo Semigroup, aggiungendo un metodo empty al combine del semigroup. Il metodo empty deve restituire un valore che se combinato con qualsiasi altra istanza di quel tipo restituisce l'altra istanza: 
\begin{minted}{scala}
(combine(x, empty) == combine(empty, x) == x)
\end{minted}
\noindent Infatti, dato che i monoidi estendono i semigroup hanno anche essi l’associativity, avendo
però un valore empty hanno anche l’identity definita come: combine(x, empty) = combine(empty, x) = x.
Ad esempio:
\begin{minted}{scala}
val empty = 0
combine(1, empty) == combine(empty, 1)
=> true
combine(1, empty) == 1
=> true
\end{minted}

\subsection{Funtori}
Si definisce come: 
\begin{minted}{scala}
trait Functor[F[_]] {
  def map[A, B](fa: F[A])(f: A => B): F[B]
}
\end{minted}
\noindent Un Functor è una classe di tipo onnipresente che coinvolge tipi che hanno un "foro", cioè tipi che hanno la forma F[*], come Option, List e Future. (Questo è in contrasto con un tipo come Int che non ha foro, o Tuple2 che ha due fori (Tuple2[*,*])). Il Funtore prevede una singola operazione, denominata map.

\noindent Per quanto riguarda i funtori sono presenti due laws: Composition e Identity.
\subsubsection{Composition}
La compositione si definisce come:
\begin{minted}{scala}
 x.map(f).map(g) = x.map(f.andThen(g))
 \end{minted}
\noindent Ovvero mappare con una funzione f poi con una funzione g è l'equivalente di mappare per composizione
con f e g. Il seguente codice mostra un esempio di questa equivalenza:
\begin{minted}{scala}
val v = List(1, 1)
val g: Int => Int = _ + 1
=> g: Int => Int = <function1>
val f: Int => Int = _ - 1
=> f: Int => Int = <function1>
val l0 = v.map(f).map(g)
=> List(1, 1)
val l1 = v.map(f.andThen(g))
=> List(1, 1)
l0 == l1
=> true
 \end{minted}

\subsubsection{Identity}
L'identità si definisce come:
\begin{minted}{scala}
v.map(x => x) = v
\end{minted}
\noindent Questa equivalenza ci dice che la map applicata ad un oggetto con la sua identità restituisce l'oggetto stesso. Infatti:
\begin{minted}{scala}
val v0 = List(1, 1)
val v1 = v0.map(x => x)
=> List(1, 2)
v0 == v1
=> true
 \end{minted}

\subsection{Apply}
Apply estende la classe di tipo Functor (che presenta la funzione map) con una nuova funzione ap. La funzione ap è simile alla map in quanto stiamo trasformando un valore in un contesto; un contesto è F in F[A]; un contesto può essere Option, List or Future per esempio. Tuttavia, la differenza tra ap e map è che per ap la funzione che si occupa della trasformazione è di tipo F[A $\Rightarrow$ B], mentre per la mappa è A $\Rightarrow$ B:
\begin{minted}{scala}
trait Functor[F[_]] {
  def map[A, B](fa: F[A])(f: A => B): F[B]
}

trait Apply[F[_]] extends Functor[F]{
    def ap[A, B](ff: F[(A) => B])(fa: F[A]): F[B]
}
\end{minted}
Ecco le implementazioni di Apply per i tipi Option e List:
\begin{minted}{scala}
import cats._

implicit val optionApply: Apply[Option] = new Apply[Option] {
  def ap[A, B](f: Option[A => B])(fa: Option[A]): Option[B] =
    fa.flatMap(a => f.map(ff => ff(a)))

  def map[A, B](fa: Option[A])(f: A => B): Option[B] = fa map f

}

implicit val listApply: Apply[List] = new Apply[List] {
  def ap[A, B](f: List[A => B])(fa: List[A]): List[B] =
    fa.flatMap(a => f.map(ff => ff(a)))

  def map[A, B](fa: List[A])(f: A => B): List[B] = fa map f

}
\end{minted}

\noindent Infatti:
\begin{minted}{scala}
val intToString: Int => String = _.toString
intToString: Int => String = <function1>
val v = Apply[Option].map(Some(1))(intToString) 
=> Some("1")
    
\end{minted}
\subsection{Applicative}
Si definiscono come:

\begin{minted}{scala}
trait Applicative[F[_]] extends Apply[F]{
  def ap[A, B](ff: F[A => B])(fa: F[A]): F[B]
  def pure[A](a: A): F[A]
  def map[A, B](fa: F[A])(f: A => B): F[B] = ap(pure(f))(fa)
}
\end{minted}

\noindent Le Applicative estendono Apply aggiungendo un singolo metodo chiamato pure, pure eleva qualsiasi valore nel Funtore Applicativo in pratica questo metodo prende qualsiasi valore e restituisce il valore nel contesto del funtore. Ad esempio per Option potrebbe essere Some(\_), per Future Future.successful e per List il singleton list.

\noindent Quando si parla di applicative la documentazione afferma che ap è un po' complicato da spiegare e motivare, quindi esamineremo una formulazione alternativa ma equivalente tramite product e definisce un Applicativa come:
\begin{minted}{scala}
trait Applicative[F[_]] extends Functor[F] {
  def product[A, B](fa: F[A], fb: F[B]): F[(A, B)]

  def pure[A](a: A): F[A]
}
\end{minted}

\noindent Per quanto riguarda le applicative sono presenti tre laws: Associativity, Left Identity e Right Identity.

\subsubsection{Associativity}
L'associatività afferma che: indipendentemente dall'ordine in cui si sommano tre valori, il risultato è isomorfo.
Infatti:
\begin{itemize}
    \item fa.product(fb).product(fc) $\sim$ fa.product(fb.product(fc))
\end{itemize}
\noindent Con map, questo può essere trasformato in un'uguaglianza con:
\begin{itemize}
    \item fa.product(fb).product(fc) = fa.product(fb.product(fc)).map { case (a, (b, c)) $\Rightarrow$ ((a, b), c) }
\end{itemize}

\subsubsection{Left Identity}
\noindent Left Identity afferma che: comprimendo un valore a sinistra con unità si ottiene qualcosa di isomorfo al valore originale.
Infatti:
\begin{itemize}
    \item pure(()).product(fa) $\sim$ fa
\end{itemize}

\noindent Come uguaglianza:
\begin{itemize}
    \item pure(()).product(fa).map(\_.\_2) = fa
\end{itemize}

\subsubsection{Right Identity}
Right Identity afferma che: comprimere un valore a destra con l'unità risulta in qualcosa di isomorfo al valore originale.
Infatti:
\begin{itemize}
    \item fa.product(pure(())) $\sim$ fa
\end{itemize}
Come uguaglianza:
\begin{itemize}
    \item fa.product(pure(())).map(\_.\_1) = fa
\end{itemize}




\subsection{Monadi}
Si definisce come:
\begin{minted}{scala}
trait Monad[F[_]] extends FlatMap[F] with Applicative[F]{
        def flatten[A](ffa: F[F[A]]): F[A]
}
\end{minted}
\noindent Monad estende la Applicative con una nuova funzione flatten. Flatten prende un valore in un contesto annidato (es. F[F[A]]dove F è il contesto) e "unisce" i contesti insieme in modo da avere un unico contesto (es. F[A]).
Ad esempio:
\begin{minted}{scala}
Option(Option(1)).flatten
// res0: Option[Int] = Some(value = 1)
Option(None).flatten
// res1: Option[Nothing] = None
List(List(1),List(2,3)).flatten
// res2: List[Int] = List(1, 2, 3)    
\end{minted}
La seguente è una definizione più completa dell'interfaccia:
\begin{minted}{scala}
trait Monad[F[_]] extends FlatMap[F] with Applicative[F]{
    override def map[A, B](fa: F[A])(f: A => B): F[B] = flatMap(fa)(a => pure(f(a)))
    def flatten[A](ffa: F[F[A]]): F[A]
    def flatMap[A, B](fa: F[A])(f: (A) => F[B]): F[B]
    def pure[A](x: A): F[A]
    def tailRecM[A, B](a: A)(f: (A) => F[Either[A, B]]): F[B] 
    //Keeps calling f until a scala.util.Right[B] is returned.

}
\end{minted}
\noindent Se Applicative è già definita e flatten si comporta bene, estendere la Applicative a Monad è banale. Per fornire la prova che un tipo appartiene alla Monad classe type, l'implementazione di cats ci richiede di fornire un'implementazione di pure(che può essere riutilizzata da Applicative) e flatMap. Possiamo usare flatten per definire flatMap: flatMap è una map seguito da flatten. Al contrario, flatten si ottiene solo usando flatMap e la funzione di identità x $\Rightarrow$ x. Ad esempio flatMap(\_)(x $\Rightarrow$ x).

\noindent flatMap è spesso considerata la funzione principale delle Monadi in Cats che segue questa tradizione fornendo implementazioni flatten e map derivati da flatMap e pure. La ragione di questo è che il nome flatMap ha un significato speciale in scala, in quanto le for-comprehension(sequenza di chiamate di uno o più metodi foreach, map, flatMap, ecc..) si basano su questo metodo per concatenare insieme le operazioni in un contesto di monadi.

\subsubsection{tailRecM}
Oltre a richiedere flatMap e pure, le Monadi in Cats richiedono anche tailRecM che codifica la ricorsione monadica stack safe, come descritto in Stack Safety for Free di Phil Freeman. Poiché la ricorsione monadica è comune nella programmazione funzionale ma non è sicura nello stack sulla JVM, Cats ha scelto di richiedere questo metodo per tutte le implementazioni della monade invece che solo un sottoinsieme. Tutte le funzioni che richiedono la ricorsione monadica in Cats lo fanno tramite tailRecM.

Un esempio di implementazione di Monad per Option è mostrato qui sotto. Nota la coda ricorsiva e quindi l'implementazione sicura di tailRecM.
\begin{minted}{Scala}
import cats.Monad
import scala.annotation.tailrec

implicit val optionMonad = new Monad[Option] {
  def flatMap[A, B](fa: Option[A])(f: A => Option[B]): Option[B] = fa.flatMap(f)
  def pure[A](a: A): Option[A] = Some(a)

  @tailrec
  def tailRecM[A, B](a: A)(f: A => Option[Either[A, B]]): Option[B] = f(a) match {
    case None              => None
    case Some(Left(nextA)) => tailRecM(nextA)(f) // continue the recursion
    case Some(Right(b))    => Some(b)            // recursion done
  }
}
\end{minted}

\subsubsection{FlatMap}

FlatMap è una classe di tipo strettamente correlata è identica a Monad, meno il metodo pure. Infatti in Cats Monad c'è una sottoclasse di FlatMap(da cui ottiene flatMap) e Applicative (da cui ottiene pure).
\begin{minted}{Scala}
trait FlatMap[F[_]] extends Apply[F] {
  def flatMap[A, B](fa: F[A])(f: A => F[B]): F[B]
}

trait Monad[F[_]] extends FlatMap[F] with Applicative[F]
\end{minted}

\noindent Le law per FlatMap sono solo le leggi di Monad che non menzionano pure.Infatti anche essa è Associativa, Left Identity, Right Identity. Una delle motivazioni per l'esistenza di FlatMap è che alcuni tipi di istanze hanno FlatMap ma non Monad. Un esempio è Map[K, *]. Considera il comportamento di pure per Map[K, A]. Dato un valore di type A, abbiamo bisogno di associarvi qualche arbitrario K ma non abbiamo modo di farlo. Tuttavia, dato che esiste Map[K, A]e Map[K, B](o Map[K, A $\Rightarrow$ B]), è semplice accoppiare (o applicare funzioni a) valori con la stessa chiave. Quindi Map[K, *]ha un'istanza FlatMap.



\section{Typeclass}
Una Typeclass è un potente strumento utilizzato nella programmazione funzionale per abilitare il polimorfismo ad hoc, più comunemente noto come overloading. Laddove molti linguaggi orientati agli oggetti sfruttano l'ereditarietà per il codice polimorfico, la programmazione funzionale tende verso una combinazione di polimorfismo parametrico (pensa parametri di tipo, come i generici Java) e polimorfismo ad hoc.

\noindent Prendiamo come esempio quello di voler costruire un meccanismo di per il calcolo dell'area di alcune forme geometriche, avremo quindi le seguenti classi:
\begin{minted}{scala}
// definizione delle case class per le forme geometriche
case class Circle(radius: Double)
case class Square(side: Double)
case class Rectangle(width: Double, height: Double)
    
\end{minted}

\noindent Ora si può definire la typeclass Area relativa al calcolo dell'area come un trait:

\begin{minted}{scala}
// definizione della typeclass Area
trait Area[A] {
  def getArea(a: A): Double
}
\end{minted}

\noindent La typeclass sopra è parametricamente polimorfica perchè viene definita con un tipo \textbf{A}, il che significa che si può sostituire A con qualsiasi tipo. Un trait da solo o usato in altro modo non si qualifica come typeclass. Definiamo ora la funzione area che vogliamo rendere rendere poliformica ad-hoc:
\begin{minted}{scala}
// definizione della funzione per calcolare l'area polimorfica ad hoc
def area[A](a: A)(implicit area: Area[A]): Double = area.getArea(a)
\end{minted}

\noindent Il primo parametro è di un tipo \textbf{A}, il secondo parametro richiede di definire la type variable per il tipo A che dovrebbe essere un sottotipo della typeclass Area definita precedentemente. Definiamo quindi i seguenti sottotipi:

\begin{minted}{scala}
// definizione delle istanze della typeclass Area per le forme geometriche

implicit val circleArea: Area[Circle] = new Area[Circle] {
  def getArea(c: Circle): Double = math.Pi * c.radius * c.radius
}

implicit val squareArea: Area[Square] = new Area[Square] {
  def getArea(s: Square): Double = s.side * s.side
}

implicit val rectangleArea: Area[Rectangle] = new Area[Rectangle] {
  def getArea(r: Rectangle): Double = r.width * r.height
}
\end{minted}

\noindent In questo modo si aumenta il range di tipi che il metodo area può gestire, creando il polimorfismo ad-hoc. Ciò è molto importante in quanto il concetto di typeclass è alla base di Cats e Cats-Effect.

\section{Caratteristiche e struttura}
Cats è una libreria che fornisce astrazioni per la programmazione funzionale nel linguaggio di programmazione Scala. 

\subsection{Perchè Cats}
Scala è un linguaggio con un approccio ibrido che supporta sia la programmazione orientata agli oggetti che quella funzionale il che non lo rende un linguaggio puramente funzionale. La libreria Cats si sforza di fornire astrazioni di programmazione puramente funzionale che siano soprattutto efficienti ed efficaci. L’obiettivo di Cats è fornire una base per un’ecosistema di librerie pure e tipizzate per supportare la programmazione funzionale in applicazioni Scala.

\subsection{Caratteristiche}
In primo luogo, Cats è molto accessibile a nuovi sviluppatori grazie al lavoro svolto per renderla intuitiva e user-friendly. Inoltre, la libreria è caratterizzata da:

\begin{itemize}
    \item  \textbf{Accessibilità:} uno dei principi della libreria è quello legato alla trasmissione dei concetti della stessa, la raccolta di successi e fallimenti in questo processo ha portato Cats ad essere e voler continuare ad essere una libreria accessibile e di facile approccio per i  nuovi sviluppatori.
    \item \textbf{Minimalità:} la libreria punta ad essere modulare. Per farlo punta su nucleo compatto che contiene solo le typeclasses, il minimo indispensabile di strutture dati necessarie per supportarle e istanze di typeclasses per tali strutture dati e tipi di libreria standard.
    \item \textbf{Documentazione:} il team ritiene che avere molta documentazione sia un obiettivo molto importante in quanto ritenuto un grande passo verso il loro obiettivo di accessibilità. C'è anche un grande sforzo nel documentare molto e in modo chiaro il codice con esempi.
    \item \textbf{Efficienza:} un punto chiave in Cats a cui il progetto tiene molto: mantenere la libreria il più efficiente possibile senza fare a meno di purezza e usabilità. Laddove è necessario trovare il compromesso tra questi due aspetti a volte in contrasto, l'obbiettivo è renderlo evidente e ben documentato.
    
\end{itemize}

\subsection{Struttura}
La libreria Cats è molto consistente e mette a disposizione molti moduli ricchi di strutture dati e funzionalità, i due moduli di base richiesti sono i seguenti:
\begin{itemize}
    \item \textbf{cats-kernel:} contiene un piccolo insieme di typeclasses
    \item  \textbf{cats-core:} contiene la maggior parte delle typeclasses e delle funzionalità core.
\end{itemize}

Oltre a questi due moduli la libreria si compone di: 
\begin{itemize}
    \item \textbf{cats-laws:} Leggi per il test delle typeclass e delle istanze.
    \item \textbf{cats-free:} Strutture libere come la monade libera e supporto alle typeclass.
    \item \textbf{cats-testkit} Libreria per scriere test sulle istanze typeclass  e sulle laws.
    \item \textbf{algebra:} Typeclasses che rappresentano strutture algebriche.
    \item \textbf{alleycats-core:} istanze e classi di senza laws.
\end{itemize}

Esistono poi altri moduli separati da quelli contenuti nel repository principale di Cats, tra cui Cats-Effect su cui ci si soffermerà particolarmente nel progetto per la gestione della concorrenza, delle risorse e I/O.



