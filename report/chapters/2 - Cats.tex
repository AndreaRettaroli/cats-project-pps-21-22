\chapter{Cats}

\section{Cats}
La libreria Cats è nata nel 2014, grazie all'iniziativa di un gruppo di sviluppatori di software funzionale, tra cui Michael Pilquist, Travis Brown, Lars Hupel e altri. Essi hanno avvertito la necessità di creare una libreria che potesse fornire un'implementazione consistente e componibile dei concetti fondamentali della programmazione funzionale.

\noindent Cats è stata ispirata dalla libreria Scalaz, una libreria di supporto per la programmazione funzionale in Scala, sviluppata da Tony Morris e dagli altri membri della comunità Scalaz. Tuttavia, rispetto a Scalaz, Cats è stata progettata con un'attenzione maggiore alla modularità e alla compatibilità con altre librerie di Scala.

\noindent Il nome "Cats" è un acronimo di "Category Theory Scala", in riferimento alla teoria delle categorie, una branca della matematica che fornisce un linguaggio formale per la descrizione di concetti astratti e relazioni tra di essi. La teoria delle categorie è stata una fonte di ispirazione per la progettazione di Cats, in quanto essa offre un'astrazione potente e componibile per i concetti fondamentali della programmazione funzionale.

\noindent Negli anni successivi alla sua nascita, Cats ha avuto un crescente successo all'interno della comunità di sviluppatori Scala, diventando una delle librerie più utilizzate per la programmazione funzionale. Grazie alla sua architettura modulare e alla compatibilità con altre librerie di Scala, essa ha consentito lo sviluppo di applicazioni altamente performanti, robuste e scalabili, permettendo di sfruttare al massimo le potenzialità della programmazione funzionale.

\noindent Ad oggi, Cats è una libreria molto utilizzata dalle grandi aziende di sviluppo software che adottano la programmazione funzionale in Scala. Ad esempio, Twitter utilizza Cats all'interno della propria infrastruttura di servizi, sfruttando le funzionalità di concorrenza e di gestione delle risorse per sviluppare applicazioni altamente performanti e scalabili. In particolare, Twitter ha sviluppato una libreria di supporto chiamata "Finagle", che si basa su Cats e su altre librerie di Scala, per fornire un'architettura di servizi distribuiti altamente efficiente e affidabile.

\noindent Anche la società di consulenza tecnologica Netflix utilizza Cats all'interno della propria piattaforma di streaming video, sfruttando le funzionalità di gestione delle risorse e di concorrenza per garantire prestazioni elevate e stabili.

\noindent Oltre alle grandi aziende, anche molte startup e imprese di medie dimensioni utilizzano Cats per lo sviluppo di applicazioni web, servizi backend e strumenti di analisi dati. Grazie alla sua architettura modulare e alla flessibilità, Cats è in grado di soddisfare le esigenze di una vasta gamma di applicazioni e di ambienti di sviluppo.

\noindent Essa fornisce un'ampia gamma di funzionalità e costrutti, che consentono di scrivere codice più conciso, espressivo e robusto permettendo astrazioni per la programmazione funzionale fornendo supporti per la gestione dell'I/O e della concorrenza. Una delle funzionalità principali di Cats è la gestione degli side-effect e delle operazioni asincrone, che è ottenuta attraverso l'utilizzo di tipi di dati funzionali come i Funtori, i Monoidi, le Applicative che approfondiremo in seguito. Questi tipi di dati consentono di gestire i side-effect e le operazioni asincrone in modo sicuro e componibile. Cats offre una serie di costrutti e funzionalità utili per la gestione delle collezioni di dati, tra cui mappe, insiemi, sequenze e stream. Questi costrutti consentono di manipolare le collezioni di dati in modo elegante e funzionale, fornendo numerosi metodi e funzioni per la trasformazione, l'ordinamento e l'aggregazione dei dati. Inoltre, fornisce anche una serie di funzionalità per la gestione delle eccezioni, compresa la gestione degli errori e la loro propagazione in modo sicuro e componibile. La libreria offre un supporto completo per la programmazione generica, consentendo di scrivere codice parametrico e riutilizzabile. Cats contiene un'ampia gamma di funzionalità per la programmazione asincrona, tra cui il supporto per la gestione degli eventi, la gestione delle risorse e la concorrenza. Queste funzionalità consentono di scrivere applicazioni altamente performanti e scalabili, sfruttando al massimo le capacità di Scala e della programmazione funzionale.

\noindent In sintesi, la libreria Cats offre una vasta gamma di funzionalità e costrutti, che consentono di scrivere codice più conciso, espressivo e robusto. Grazie alla sua architettura funzionale, essa è particolarmente adatta per lo sviluppo di applicazioni asincrone, distribuite e altamente concorrenti.

\noindent Le seguenti sezioni mostrano i costrutti matematici di programmazione funzionale forniti da Cats.

\section{Costrutti Matematici}

\subsection{Semigroup}

\subsection{Monoidi}

\subsection{Funtori}

\subsection{Composition}

\subsection{Identity}

\subsection{Monadi}

\subsection{Left Identity}

\subsection{Right Identity}

\subsection{Associativity}
